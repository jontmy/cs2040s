\section{Union-Find}
\emph{An abstract data type for disjoint sets and connected components.}

\code{union} combines two objects, and \code{find} checks whether there is a path between two objects.

A naive implementation associates each object with an identifier.
\code{find} checks whether their identifiers are the same in $O(1)$.
\code{union} has to iterate over all identifiers and combine them in $O(n)$.

A better implementation associates each object with a parent.
\code{find} now checks whether they have the same highest common ancestor (HCA) in $O(n)$.
\code{union} has to find both their HCAs to link together in $O(n)$.

An optimization using \textbf{weighted union} associates a tree size with each object,
so \code{union} links the smaller tree to the larger one in $O(\log n)$.
Since the heights of the trees of size $n$ is at most $\log(n)$, \code{find} takes $O(\log n)$ time.

The final optimization adds \textbf{path compression}.
After each time the root is found, every node on the path is linked to the root.
Both \code{union} and \code{find} take $O(\alpha(m, n))$ amortized time each for $m$ operations,
and $O(\log n)$ in the worst case.

With path compression but without weighted union, both \code{union} and \code{find} take $O(\log n)$ time.