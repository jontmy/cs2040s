\section{Searching}

\subsection{Binary Search}
\emph{Find the index of an element in a sorted array if it exists.}

\begin{lstlisting}[
    mathescape,
    columns=fullflexible,
    basicstyle=\fontfamily{lmvtt}\selectfont,
  ]
int binary_search(A, key, n):
    lo = 0
    hi = n - 1
    while lo < hi:
        mid = lo + (hi - lo) / 2
        if key $\leq$ A[mid]: hi = mid
        else: lo = mid + 1
    return A[lo] == key ? lo : -1 
\end{lstlisting}

$\code{A[lo]} \leq \code{key} \leq \code{A[hi]}$ is a \textbf{loop invariant},
and on iteration $k$, $\code{hi} - \code{lo} \leq \frac{\code{n}}{2^\code{k}}$.

This has a time complexity of $O(\log n)$.

\subsection{Peak Finding}
\emph{Find the index of a local maximum in an unsorted array.}

A \textbf{peak} is a local maximum in \code{A} such that $\code{A[i]} \geq \code{A[i-1]}$ and $\code{A[i]} \geq \code{A[i+1]}$.
We also assume that $\code{A[-1]} = \code{A[n]} = \code{-MAX-INT}$.

\begin{lstlisting}[
    mathescape,
    columns=fullflexible,
    basicstyle=\fontfamily{lmvtt}\selectfont,
  ]
int find_peak(A, key, n):
    if A[n/2 + 1] > A[n/2]:
        return find_peak(A[n/2 + 1 : n], n/2)
    else if A[n/2 - 1] > A[n/2]:
        return find_peak(A[1 : n/2 - 1], n/2)
    else A[n/2] is a peak:
        return n/2
\end{lstlisting}

There will always exist a peak in \code{A[lo:hi]}, and every peak in \code{A[lo:hi]} is a peak in \code{A[1:n]}.

The recurrence relation is $T(n) = T(\frac{n}{2}) + \theta(1)$ which gives a time complexity of $O(\log n)$.

\subsection{Quick Select}
\emph{Find the k-th smallest element in an unsorted array.}

Instead of sorting the entire array, we only do the minimum amount of sorting needed by recursing
once on the correct side of a pivot. Recursing on both sides is Quick Sort!

\begin{lstlisting}[
    mathescape,
    columns=fullflexible,
    basicstyle=\fontfamily{lmvtt}\selectfont,
  ]
quick_select(A, l, r, k):
  loop
      if l = r: return A[left]
      pivot = partition(A, l, r, random(l, r))
      if k = pivot: return A[k]
      else if k < pivot: r = pivot - 1
      else: l = p + 1
\end{lstlisting}

This has a time complexity of $O(n)$.