\section{Binary Heaps}
A \textbf{complete binary tree} is a binary tree where there are $2^i$ nodes at depth $i$
except the last level which is \textbf{left-justified}.

A binary heap is implemented with an \emph{implicit data structure} without pointers,
using a \textbf{heap array}.

To access children of node $i$, we use index arithmetic:
the left child is at index $2i+1$,
and the right child is at index $2i+2$.
To access the parent: $\lfloor \frac{i-1}{2} \rfloor$.

The \textbf{Max-Heap property} states that the key of every vertex is greater than or equal to the values of its children.

\code{insert} appends a new node to the heap array, 
and calls \code{swim} to swap the node upward with its ancestors until it is in the correct position.

\code{delete-max} swaps the last node in the heap array with the root, 
and calls \code{sink} to swap the root downward with the larger of its descendants until it is in the correct position.

Both \code{insert} and \code{delete-max} are $O(\log n)$ time.

\code{build} can be done in $O(n)$ time if we start from the end of the array and only use \code{sink}.

\subsection{Priority Queue}
\emph{An abstract data structure is a queue with elements served in order of their assigned priorities.}

A binary heap is an implementation for a priority queue which must be a complete binary tree and 
obeys the MaxHeap property.