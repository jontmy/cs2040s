\section{Minimum Spanning Trees}
\emph{An acyclic subset of edges connecting all vertices, with minimum total weight.}

MSTs cannot be used to find shortest paths.

A \textbf{cut} of a graph partitions the vertices into two disjoint subsets.
An edge crosses a cut if it has one vertex in each of these two subsets.

MSTs have four properties:
\begin{enumerate}
    \item MSTs are acyclic.
    \item If an MST is cut, both subtrees are MSTs.
    \item For every cycle, the maximum weight edge is not in the MST, but the minimum weight edge may or may not be.
    \item For every vertex (and cut), the minimum weight edge is in the MST.
\end{enumerate}

\subsection{Prim's Algorithm}
We maintain a set $S$ for the vertices in the MST,
a priority queue $Q$ of vertices ordered by their distance $d$ from the growing MST,
as well as a parent pointer $\pi(v)$ for each vertex $v$.

We initialize $d(v)$ to $\infty$ for every vertex $v$,
and start at some arbitrary vertex $s$ such that $S = \{s\}$ and $d(s) = 0$.

Then, we repeatedly dequeue the vertex $u$ with the smallest distance $d$,
removing it from $Q$ and adding it to $S$.

For every edge $(u, v)$ in the graph, we relax it as so:
\begin{lstlisting}[
    mathescape,
    columns=fullflexible,
    basicstyle=\fontfamily{lmvtt}\selectfont,
  ]
relax(u, v):
    w = weight(u, v)
    if d(v) $>$ w:
        d(v) = Q.decrease_key(u, w) // O(log |V|)
        $\pi$(v) = u
\end{lstlisting}

This works because each added edge is the minimum across some cut,
therefore each edge is in the MST.

Every vertex is only added and removed once from $Q$, taking $O(|V| \log |V|)$ in total.

Each edge is only relaxed once (calling \code{decrease-key}), taking $O(|E| \log |V|)$ in total.

Therefore, this has a time complexity of \textbf{O(E log V)}.

\subsection{Kruskal's Algorithm}
We will use union-find to keep track of the vertices in the MST.

We sort the edges by weight in ascending order and iterating through them.
If the edge connects two vertices in the same connected component (with \code{find}), we ignore it.
Otherwise, we add it to the MST and call \code{union}.

This works because the ignored edge would have been the heaviest edge on the cycle,
which violates the MST property.
Also, all other lighter edges have already been considered.

This has a time complexity of \textbf{O(E log V)} since $\log |E| = O(\log |V|^2) = O(2 \log |V|) = O(\log |V|)$.

With integer weights, $O(n)$ counting sort may be used such that the time complexity is improved to $O(|E|)$.
With a $O(n^2)$ sort, the time complexity is $O(|E|^2)$.

\subsection{Variants}
If all edge weights are equal, any spanning tree found by DFS or BFS will be an MST since
both the spanning tree an MSTs have exactly $V - 1$ edges.

Reweighting a graph has no effect on the MST.

A directed MST is a much harder problem, but in the special case with one root, an MST is
the tree of all edges with the minimum weight, which takes $O(E)$ time.

A maximum spanning tree can be found by negating each edge weight, or by running Kruskal's in reverse.