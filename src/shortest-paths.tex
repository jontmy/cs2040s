\section{Shortest Paths}

\subsection{Bellman-Ford Algorithm}
\emph{Computes the shortest path for graphs with negative weights.
Can be used to detect negative weight cycles.}

Let $\delta(u, v)$ be the weight of the shortest path from $u$ to $v$.
By the triangle inequality, $\delta(s, v) \leq \delta(s, u) + \delta(u, v)$.

For each vertex $v$, we maintain an estimate $d(v)$ of the distance from a source $s$, and a parent pointer $\pi(v)$.
All estimates are initialized to $\infty$, and $d(s) = 0$.

The algorithm performs $|V| - 1$ iterations over the graph, during which each edge is relaxed:

\begin{lstlisting}[
    mathescape,
    columns=fullflexible,
    basicstyle=\fontfamily{lmvtt}\selectfont,
  ]
relax(u, v):
    if d(v) $>$ d(u) + weight(u, v):
        d(v) = d(u) + weight(u, v)
        $\pi$(v) = u
\end{lstlisting}

Relaxation mainatains the invariant $d(v) \geq \delta(s, v) \; \forall v \in V$, i.e., we never under-estimate.

If an entire iteration does not change any estimates, then we can terminate early.
After $n$ iterations, the $n$-hop estimate on the shortest path (not every path!) is correct, i.e., $d(v_n) = \delta(s, v_n)$

Also, if $P$ is the shortest path from $s$ to $v$, and if $P$ goes through $u$, then $P$ is also the shortest
path from $s$ to $u$.

If we run a $|V|$-th iteration and the estimates still change, then we have found a negative weight cycle.

This algorithm has a time complexity of \textbf{O(EV)}.

\subsection{Dijkstra's Algorithm}
\emph{Computes the shortest path for graphs with only non-negative weights, including cyclic graphs.}

For each vertex $v$, we maintain the distance $d(v)$ from a source $s$, and a parent pointer $\pi(v)$.

We also maintain a priority queue $Q$ of vertices ordered by their distance $s$, i.e. the $d$-values.
$d(s)$ is initialized to 0.

\begin{lstlisting}[
    mathescape,
    columns=fullflexible,
    basicstyle=\fontfamily{lmvtt}\selectfont,
  ]
dijkstra():
    initialize()  // O(|V| log |V|)
    while Q is not empty: // O(|V|)
        u = extract_min(Q)  // O(log |V|)
        for each edge (u, v):  // O(|E|)
            relax(u, v)  // O(log |V|)
\end{lstlisting}

The relax operation is the same as in the Bellman-Ford algorithm, 
except that \code{decrease-key} is used to update the $d$-values.

The time complexity of this algorithm with an AVL tree is \textbf{O(E log V)},
but \textbf{O(E + V log V)} with a Fibonacci heap.
The complexities are simplified as $|E| = \theta(|V|^2)$.

Graphs can be reweighted by multiplying the edge weights by a constant factor,
but addition by a constant factor will break Dijkstra's algorithm.

We can also find the longest path by negating the edge weights, as long as the graph is acyclic.

If we wanted to find the path with shortest weight products rather than sum,
it is easier to take the logarithm of the weights, rather than modify the relax operation.

\subsection{Relaxation in Topological Order}
\emph{Computes the shortest path for directed acyclic graphs.}

Since the path is relaxed in order, the distance is correct after relaxation.

This has a time complexity of \textbf{O(V + E)}.