\section{Dynamic Programming}
\emph{Construct an optimal solution to a problem from optimal solutions to smaller subproblems.}

\subsection{SRTBOT}
\begin{enumerate}
    \item \textbf{S}ubproblem definition.
    \item \textbf{R}elate subproblem solutions recursively.
    \item \textbf{T}opological order on subproblems (DAG).
    \item \textbf{B}ase case(s) for subproblems.
    \item \textbf{O}riginal problem, solved via subproblems.
    \item \textbf{T}ime complexity.
\end{enumerate}

Use \textbf{memoization} to re-use solutions to subproblems with
a dictionary mapping subproblems to their solutions.

Recursive function computes the solution if and only if it is not already stored.

If the the input is some arbitrary sequence $x$ of length $n$, good subproblems to consider are:
\begin{enumerate}
    \item the $\theta(n)$ \textbf{prefixes} $x[:i]$,
    \item the $\theta(n)$ \textbf{suffixes} $x[i:]$, and
    \item the $\theta(n^2)$ \textbf{substrings} $x[i:j]$.
\end{enumerate}

The use of subsequences as subproblems, where some elements can be omitted,
is undesirable since there would be an exponential number of them.

\subsection{Longest Common Subsequence}
\emph{Given two subsequences $A$ and $B$, find the longest common subsequence $L$ of $A$ and $B$.}

When there are \textbf{multiple inputs}, subproblems can be generated by taking the cross product
of the subproblem spaces.

\textbf{S}: $L(i, \; j) = \operatorname{LCS}(A[i:], \; B[j:]) \; \forall i \in [0, |A|] \; \forall j \in [0, |B|]$\\[-0.2em]
\[ \textbf{R}\text{: } L(i, \; j) = \begin{cases} 
    1 + L(i + 1, \; j + 1) & \text{if} A[i] = B[j] \\
    \max\{L(i + 1, \; j), \; L(i, \; j + 1)\} & \text{otherwise}
 \end{cases}
\]

\textbf{T}: for $i = |A| \dots 0$ for $j = |B| \dots 0$

\textbf{B}: $L(|A|, \; j) = L(i, \; |B|) = 0$

\textbf{O}: $L(0, 0)$

\textbf{T}: $\theta(|A| \cdot |B|) \text{ subproblems} \cdot \theta(1) \text{ each} = \theta(|A| \cdot |B|)$

We can use \textbf{parent pointers} to reconstruct the solution.

\subsection{Longest Increasing Subsequence}
\emph{Given a sequence $A$, find the longest strictly increasing subsequence $L$ of $A$. Generalizes to non-strict as well.}

\textbf{S}: $L(i) = \operatorname{LIS}(A[i:])$ that starts with $A[i]$ (\textbf{constraint}).

\textbf{R}: $L(i) = 1 + \max \{ L(j) \; | \; i < j < |A|, \; A[i] < A[j] \} \cup \{ 0 \}$

\textbf{T}: for $i = |A| \dots 0$

\textbf{B}: $L(|A|) = 0$

\textbf{O}: $\max \{ L(i) \; | \; 0 \leq i \leq |A| \}$

\textbf{T}: $\theta(|A|) \text{ subproblems} \cdot \theta(|A|) \text{ choices} + \theta(|A|) = \theta(|A|^2)$

\subsection{All-Pairs Shortest Path (APSP)}
\emph{Find the shortest path between all pairs of vertices in a graph, using the Floyd-Warshall algorithm.}

First, number every vertex from 1 to $|V|$.

\textbf{S}: $d(u, v, k) =$ weight of the shortest path from $u$ to $v$ using only the vertices $\in \{ u, v \} \cup \{ 1, 2, \dots, k \} \linebreak \;
\forall u, v \in V \; \forall k \in [0, \; |V|]$\\[-0.2em]
\[ \textbf{R}\text{: } d(u, v, k) = \min \begin{cases} 
    d(u, v, k - 1) & k \in SP \\
    d(u, k, k - 1) + d(k, v, k - 1) & k \notin SP \\
 \end{cases}
\]

\textbf{T}: increasing k for $k = 0 \dots |V|$ for $u \in V$ for $v \in V$\\[-0.2em]
\[ \textbf{B}\text{: } d(u, v, 0) = \begin{cases} 
    0 & \text{if } u = v \\
    w(u, v) & \text{if } (u, v) \in E \\
    \infty & \text{otherwise}
 \end{cases}
\]

\textbf{O}: $d(u, v, |V|)$ assuming no negative weight cycles

\textbf{T}: $\theta(|V|^3) \text{ subproblems} \cdot \theta(1) \text{ each} = \theta(|V|^3)$

\subsection{Partitioning (Rod Cutting)}
\emph{Given a rod of length $L \in \mathbb{Z}^+$ and a value $v(l)$ for a rod of length $l \in [1, \; L]$, what is the max-value partition for the rod?}

\textbf{S}: $X(l) = $ max-value partition of length $l, \; \forall l \in [0, L]$

\textbf{R}: $X(l) = \max \{ v(p) + X(l - p) \; | \; 1 \leq p \leq l\}$

\textbf{T}: increasing $l$ for $l = 0, 1 \dots, L$

\textbf{B}: $X(0) = 0$

\textbf{O}: $X(L)$

\textbf{T}: $\theta(L) \text{ subproblems} \cdot \theta(|L|) \text{ choices} = \theta(|L|^2)$

\subsection{Subset Sum}
\emph{Given a multiset $A = \{ a_0, a_1, \dots, a_{n-1} \}$ with $n$ positive integers, and a target sum $T$, does any subset $S \subseteq A$ sum to $T$?}

This is a \textbf{decision problem} where the answer is binary.

\textbf{S}: $X(i, t) = $ does any subset $S \subseteq A[i:]$ sum to $T \linebreak \forall i \in [0, n] \; \forall t \in [0, T]$?\\[-0.2em]
\[ \textbf{R}\text{: } X(i, t) = \operatorname{OR} \begin{cases} 
    X(i + 1, t) & a_i \notin S \\
    X(i + 1, t - a_i) \text{ if } a_i \leq t & a_i \in S
 \end{cases}
\]

\textbf{T}: decreasing $i$ for $i = n, n - 1, \dots 0$\\[-0.2em]
\[ \textbf{B}\text{: } X(n, t) = \begin{cases} 
    \text{true} & \text{if } t = 0 \\
    \text{false} & \text{otherwise} \\
 \end{cases}
\]

\textbf{O}: $X(0, T)$

\textbf{T}: $\theta(nT)$

This runs in \textbf{psuedo-polynomial} time.